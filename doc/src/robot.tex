\section{Поисковый робот}

\subsection*{Описание метода}
Для сбора корпуса документов был разработан поисковый робот на языке Python. Архитектура робота включает следующие компоненты:
\begin{itemize}
    \item \textbf{Управляющий конфигурационный файл} \texttt{config.yaml}, который задает стартовые URL, разрешенные домены и параметры работы.
    \item \textbf{База данных MongoDB} для хранения скачанных HTML-документов и очереди URL для обхода. Использование двух коллекций (для документов и для очереди) обеспечивает отказоустойчивость. Робот может быть остановлен и перезапущен, продолжая работу с места остановки.
    \item \textbf{Основной цикл}, который атомарно извлекает URL из очереди, скачивает страницу, сохраняет ее и извлекает новые ссылки.
    \item \textbf{Механизм <<вежливости>>} - реализована задержка между запросами для снижения нагрузки на сервер.
    \item \textbf{Детектор CAPTCHA}, который останавливает работу при обнаружении признаков защиты от автоматических запросов, чтобы избежать блокировки.
\end{itemize}

\newpage