\CWHeader{Лабораторная работа \textnumero 4 \enquote{Булев индекс}}

\section*{Описание}
\subsection*{Внутреннее представление данных}
Для хранения индекса был разработан собственный бинарный формат, состоящий из трех файлов.
\begin{itemize}
    \item \textbf{Прямой индекс (\texttt{forward.idx}):} Хранит URL и заголовок для каждого документа. Файл состоит из таблицы смещений и области данных, что обеспечивает быстрый доступ к информации по \texttt{DocID}.
    \item \textbf{Словарь (\texttt{dictionary.dat}):} Содержит лексикографически отсортированный список всех уникальных термов. Для каждого терма хранится его длина, частота встречаемости в документах (\texttt{doc\_frequency}) и смещение в файле \texttt{postings.dat}.
    \item \textbf{Списки вхождений (\texttt{postings.dat}):} Хранит последовательности \texttt{DocID} для каждого терма. Для экономии места и подготовки к сжатию хранятся не сами \texttt{DocID}, а их разницы (\texttt{d-gaps}).
\end{itemize}

\subsection*{Метод сортировки}
Для лексикографической сортировки словаря в памяти использовалась стандартная функция C \texttt{qsort}.
\begin{itemize}
    \item \textbf{Достоинства:} Реализация быстрой сортировки (в среднем O(N log N)), доступная в стандартной библиотеке C, не требует написания с нуля.
    \item \textbf{Недостатки:} Алгоритм работает только с данными, полностью помещающимися в оперативную память. Для корпусов большего размера он неприменим.
\end{itemize}

\subsection*{Результаты}
\begin{itemize}
    \item Количество термов: \textbf{1 672 481}
    \item Средняя длина терма: \textbf{14.71 символов}
    \item Скорость индексации (общая): \textbf{202.82 мс}
    \item Скорость индексации (на документ): \textbf{6.76 мс}
\end{itemize}

\textbf{Анализ производительности и масштабирования:}
Текущая реализация ограничена объемом оперативной памяти. Малый размер хэш-таблицы к длинным цепочкам коллизий, что замедляет поиск. Для ускореения можно увеличить этот размер.
При увеличении объема данных в 10-1000 раз программа может завершиться с ошибкой нехватки памяти.

\section*{Исходный код}
\href{https://github.com/tng00/information-retrieval/tree/main/lab4}{https://github.com/tng00/information-retrieval/tree/main/lab4}

\section*{Выводы}
В ходе этой лабораторной работы я реализовал ядро поисковой системы - индексатор, создающий прямой и обратный индексы. Я на практике освоил проектирование собственных бинарных форматов данных, что позволило эффективно хранить словарь и списки вхождений. Также я реализовал собственную хеш-таблицу для ускорения процесса построения индекса.
\newpage