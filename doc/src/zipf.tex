\section{Анализ распределения: Закон Ципфа}

\subsection{Графическое представление}
Для анализа частотного распределения терминов в корпусе был построен график в логарифмических координатах (Рис. 3). На график наложено эмпирическое распределение и теоретическая кривая Закона Ципфа.

\begin{figure}[h!]
    \centering
    \includegraphics[width=0.9\textwidth]{img/3.png}
    \caption{Распределение частот токенов в логарифмических координатах}
\end{figure}

\subsection{Объяснение расхождений}
График демонстрирует, что распределение в целом следует закону Ципфа, однако наблюдаются характерные отклонения от идеальной модели:
\begin{itemize}
    \item \textbf{<<Голова>> (начало графика):} Небольшое количество самых частотных терминов встречаются чаще, чем предсказывает теория. Это объясняется тематической однородностью корпуса (медицина), где ключевые термины (<<пациент>>, <<лечение>>) доминируют.
    \item \textbf{<<Тело>> (середина графика):} Эмпирическая кривая идет почти параллельно теоретической, что подтверждает общую закономерность.
    \item \textbf{<<Хвост>> (конец графика):} Наблюдается большое количество слов с очень низкой частотой, включая множество слов, встретившихся всего один раз (гапаксы). Это приводит к появлению характерных <<ступеней>> на графике и является фундаментальным свойством естественного языка.
\end{itemize}

\newpage