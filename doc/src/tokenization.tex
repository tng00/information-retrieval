\section{Токенизация}

\subsection{Правила токенизации}
Процесс разбиения текста на токены реализован на C++ в виде утилиты командной строки, работающей как фильтр. Были выработаны следующие правила:
\begin{itemize}
    \item \textbf{Приведение к нижнему регистру:} Все символы кириллицы и латиницы переводятся в нижний регистр.
    \item \textbf{Определение границ токена:} Разделителями считаются пробелы, знаки препинания и символы перевода строки. Они не включаются в состав токена.
    \item \textbf{Состав токена:} Токеном считается непрерывная последовательность букв, цифр и дефиса (если он не в начале или конце слова).
    \item \textbf{Фильтрация:} Отбрасываются токены, состоящие только из цифр или дефисов, а также токены длиной менее двух символов после всех преобразований.
\end{itemize}

\textbf{Достоинства:} Высокая скорость обработки за счет реализации на C++ и использования конечного автомата, простота правил.
\textbf{Недостатки:} Возможна неверная обработка сложносоставных слов или аббревиатур (например, <<С.-Петербург>>).

\subsection{Результаты и анализ производительности}
\begin{itemize}
    \item Количество токенов: \textbf{81 326 141}
    \item Средняя длина токена: \textbf{11.90}
    \item Скорость токенизации: \textbf{19.45 мБ/с}
\end{itemize}

\newpage
