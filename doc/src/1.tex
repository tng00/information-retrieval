\section{Добыча корпуса документов}

\subsection{Источник и характеристика корпуса}
В качестве источника данных был использован портал научных публикаций CyberLeninka (\url{cyberleninka.ru}). Сбор документов производился из двух тематических разделов: <<Клиническая медицина>> и <<Науки о здоровье>>.

При анализе структуры HTML-документов было установлено, что страницы имеют семантическую разметку с использованием атрибутов микроразметки Schema.org. Заголовок статьи располагается в теге с атрибутом \texttt{itemprop="headline"}, а основной текст - в блоке \texttt{div} с \texttt{itemprop="articleBody"}. Данный подход позволил реализовать парсер, устойчивый к изменениям в верстке страниц.

В мета-тегах страниц также присутствует дополнительная информация: автор, год публикации, название журнала и др. Эти данные могут быть использованы в дальнейшем для реализации расширенного поиска. Тексты статей имеют научный стиль и стандартную структуру.

\subsection{Анализ существующих поисковых систем}
Был проведен сравнительный анализ двух поисковых систем: встроенного поиска на сайте CyberLeninka и поиска Google с оператором \texttt{site:cyberleninka.ru}.

\newpage
\begin{figure}
\centering
\includegraphics[width=0.6\textwidth]{/home/tng00/information-retrieval/doc/img/1.png}
\caption{Запрос "использование фотодинамической диагностики" во встроенном поисковике CyberLeninka}
\end{figure}
Сниппеты во встроенном поиске включают два фрагмента текста, содержащие запрос. Ранжирование отдает приоритет статьям, где запрос встречается в заголовке. Присутствует указание автора и года публикации. 

\newpage
\begin{figure}
\centering
\includegraphics[width=0.6\textwidth]{/home/tng00/information-retrieval/doc/img/2.jpg}
\caption{Запрос "использование фотодинамической диагностики" в поисковике Google}
\end{figure}
Ранжирование, вероятно, отдает приоритет наиболее популярным статьям. Также присутствует указание автора и года публикации.


\subsection{Итоговая статистика корпуса}
По результатам работы был собран и обработан корпус из 30176 документов.
\begin{center}
\begin{tabular}{|l|l|}
\hline
\textbf{Параметр} & \textbf{Значение} \\
\hline
Источник & cyberleninka.ru \\
\hline
Количество документов & 30176 \\
\hline
Общий размер <<сырых>> HTML & 2.26 ГB \\
\hline
Общий размер очищенного текста (TXT) & 759.89 MB \\
\hline
Средний размер <<сырого>> документа & 76.67 KB \\
\hline
Средний объём текста в документе & 25.79 KB \\
\hline
\end{tabular}
\end{center}

\pagebreak

