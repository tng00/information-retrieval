\section{Нормализация: Стемминг}

\subsection{Описание метода}
Для нормализации токенов и сокращения словаря был реализован и интегрирован в токенизатор стеммер Портера для русского языка. Стемминг применяется к каждому токену после приведения к нижнему регистру и до проверки на валидность.

\subsection{Оценка качества поиска}
Поскольку полноценная поисковая система еще не реализована, оценка качества проводилась качественно, на основе анализа гипотетических запросов.

\textbf{Улучшение качества:} Стемминг значительно повышает полноту поиска. Например, запрос \texttt{[современная терапия]} после обработки превратится в \texttt{[современ терап]}. Такой запрос найдет документы, содержащие словосочетания <<современной терапии>>, <<современную терапию>> и т.д., которые не были бы найдены без нормализации.

\textbf{Ухудшение качества (потеря точности):} Агрессивный характер стемминга может приводить к ложным срабатываниям. Например, стеммер может ошибочно свести разные по смыслу слова к одной основе. Гипотетический пример: слова <<универсальный>> и <<университет>> могут быть сведены к общей основе <<универс>>, что приведет к появлению нерелевантных документов в выдаче.

\textbf{Способы улучшения:} Для решения проблемы потери точности можно использовать более сложный метод нормализации - лемматизацию, которая приводит слово к его нормальной словарной форме (лемме) с использованием словарей. Однако этот метод значительно медленнее стемминга.

\newpage