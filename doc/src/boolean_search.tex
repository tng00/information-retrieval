\CWHeader{Лабораторная работа \textnumero 5 \enquote{Булев поиск}}

\section*{Описание}

\subsection*{Скорость выполнения запросов}
Скорость измерялась на наборе из 1000 тестовых запросов разной сложности.
\begin{itemize}
    \item Средняя скорость выполнения запроса: \textbf{1339.13 мс}
\end{itemize}

\subsection*{Примеры сложных запросов}
\begin{itemize}
    \item \textbf{Запросы с большим количеством операторов OR:} Запрос вида \texttt{[результат || исследование || данные]} вызывает длительную работу, так как требует слияния нескольких очень длинных списков документов, что создает большие промежуточные результаты.
    \item \textbf{Запросы с оператором NOT от редкого слова:} Запрос \texttt{[!гапакс]} является дорогим, так как требует создания списка почти всех документов корпуса и последующего исключения из него одного элемента. В реальных системах такие запросы часто запрещают или выполняют в связке с \texttt{AND}.
\end{itemize}

\subsection*{Тестирование корректности}
Корректность поисковой выдачи проверялась в два этапа:
\begin{enumerate}
    \item \textbf{Модульное тестирование:} Были созданы небольшие, заранее определенные списки \texttt{DocID}, на которых проверялась корректность работы функций пересечения (\texttt{AND}), объединения (\texttt{OR}) и инверсии (\texttt{NOT}).
    \item \textbf{Интеграционное тестирование:} Был создан тестовый мини-корпус из 5 документов, для которого индекс был построен <<вручную>>. Затем было выполнено 10-15 булевых запросов разной сложности. Результаты, выданные программой, сравнивались с ожидаемыми, вычисленными вручную. Этот метод позволил проверить корректность всей цепочки: от парсинга запроса до выполнения операций.
\end{enumerate}
\pagebreak

\subsection*{Интерфейсная часть}
\begin{figure}[H]
    \centering
    \includegraphics[width=0.9\textwidth]{img/4.png}
    \caption{Пример запроса}
\end{figure}

\begin{figure}[H]
    \centering
    \includegraphics[width=0.9\textwidth]{img/5.png}
    \caption{Пагинация}
\end{figure}

\section*{Исходный код}
\href{https://github.com/tng00/information-retrieval/tree/main/lab5}{https://github.com/tng00/information-retrieval/tree/main/lab5}

\section*{Выводы}
В рамках данной работы была создана полноценная поисковая система с веб-интерфейсом. Я реализовал синтаксический анализатор булевых запросов с помощью алгоритма "Сортировочная станция", что позволило корректно обрабатывать скобки и приоритет операций. Эта лабораторная работа объединила все предыдущие наработки в единый продукт и дала практический опыт в проектировании клиент-серверной архитектуры для поисковых задач.

\newpage