\section*{Требования к компиляции и запуску}

\subsection*{Требования к окружению}
\begin{itemize}
    \item ОС: Linux-совместимая (например, Ubuntu в WSL)
    \item Компилятор C++: g++
    \item Интерпретатор Python 3.8+
    \item Система управления пакетами Python: pip
    \item База данных: MongoDB (запущенная через Docker)
\end{itemize}

\subsection*{Последовательность сборки и запуска}

\begin{enumerate}
    \item Установить зависимости Python из файла \texttt{requirements.txt}:
    \begin{lstlisting}[language=bash]
pip install -r requirements.txt
    \end{lstlisting}

    \item Скомпилировать C++-утилиты (токенизатор, индексатор, поисковик):
    \begin{lstlisting}[language=bash]
g++ -O2 -o tokenizer tokenizer.cpp
g++ -O2 -o indexer indexer.cpp
g++ -O2 -o searcher searcher.cpp
    \end{lstlisting}

    \item Запустить Docker-контейнер с MongoDB:
    \begin{lstlisting}[language=bash]
docker start mongo_container
    \end{lstlisting}

    \item Запустить процесс построения индекса с помощью скрипта-оркестратора:
    \begin{lstlisting}[language=bash]
./run_indexing.sh
    \end{lstlisting}

    \item Запустить веб-сервис для демонстрации поиска:
    \begin{lstlisting}[language=bash]
python3 app.py
    \end{lstlisting}

    \item Открыть в браузере адрес \texttt{http://127.0.0.1:5000}.
\end{enumerate}

\newpage